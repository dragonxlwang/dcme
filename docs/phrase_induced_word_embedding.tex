\documentclass{article} % For LaTeX2e
\usepackage{nips13submit_e,times}
\usepackage{hyperref}
\usepackage{url}
%\documentstyle[nips13submit_09,times,art10]{article} % For LaTeX 2.09
\definecolor{dgreen}{rgb}{0.00,0.39,0.00}

\title{Phrase Identification and Embedding}


%\author{
%David S.~Hippocampus\thanks{ Use footnote for providing further information
%about author (webpage, alternative address)---\emph{not} for acknowledging
%funding agencies.} \\
%Department of Computer Science\\
%Cranberry-Lemon University\\
%Pittsburgh, PA 15213 \\
%\texttt{hippo@cs.cranberry-lemon.edu} \\
%\And
%Coauthor \\
%Affiliation \\
%Address \\
%\texttt{email} \\
%\AND
%Coauthor \\
%Affiliation \\
%Address \\
%\texttt{email} \\
%\And
%Coauthor \\
%Affiliation \\
%Address \\
%\texttt{email} \\
%\And
%Coauthor \\
%Affiliation \\
%Address \\
%\texttt{email} \\
%(if needed)\\
%}

% The \author macro works with any number of authors. There are two commands
% used to separate the names and addresses of multiple authors: \And and \AND.
%
% Using \And between authors leaves it to \LaTeX{} to determine where to break
% the lines. Using \AND forces a linebreak at that point. So, if \LaTeX{}
% puts 3 of 4 authors names on the first line, and the last on the second
% line, try using \AND instead of \And before the third author name.

\newcommand{\fix}{\marginpar{FIX}}
\newcommand{\new}{\marginpar{NEW}}

\nipsfinalcopy% Uncomment for camera-ready version

\begin{document}


\maketitle

\begin{abstract}
Word based embeddings ignores the cognitive and linguistic importance of
phrases. Most techniques derive from Markovian Random Fields (MRF) that
associates text with fixed-order sequential model. We go beyond those by
incorporating the inference for word bonding and the deduced phrasal enriched
representation for text are trained for both single word and phrase embeddings.
\end{abstract}

\section{Dirichlet Word Bounding}
Our insight into the formation of phrase from language is that words are bound
together to behave like single word. Intuitively, when substituting the phrase
into an artificial word, it should yield a model which assigns higher likelihood
to the text than treating the words in the phrase separately. 
\subsubsection{References}
\bibliographystyle{apalike}
% \bibliography{/Users/xlwang/Dropbox/Reading/refs.bib}
\bibliography{refs.bib}
\end{document}
